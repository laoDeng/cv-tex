%% start of file `template-zh.tex'.
%% Copyright 2006-2013 Xavier Danaux (xdanaux@gmail.com).
%
% This work may be distributed and/or modified under the
% conditions of the LaTeX Project Public License version 1.3c,
% available at http://www.latex-project.org/lppl/.


\documentclass[11pt,a4paper,sans]{moderncv}   % possible options include font size ('10pt', '11pt' and '12pt'), paper size ('a4paper', 'letterpaper', 'a5paper', 'legalpaper', 'executivepaper' and 'landscape') and font family ('sans' and 'roman')

% moderncv 主题
\moderncvstyle{banking}                        % 选项参数是 ‘casual’, ‘classic’, ‘oldstyle’ 和 ’banking’
\moderncvcolor{green}                          % 选项参数是 ‘blue’ (默认)、‘orange’、‘green’、‘red’、‘purple’ 和 ‘grey’
%\nopagenumbers{}                             % 消除注释以取消自动页码生成功能

% 字符编码
\usepackage[utf8]{inputenc}                   % 替换你正在使用的编码
\usepackage{CJKutf8}

% 调整页面出血
\usepackage[scale=0.9]{geometry}
%\setlength{\hintscolumnwidth}{3cm}           % 如果你希望改变日期栏的宽度

% 强调用扩展包
\usepackage{ulem}

% 个人信息
\name{陈}{登}
%\title{简历题目 (可选项)}                     % 可选项、如不需要可删除本行
%\address{中国地质大学(武汉)}{}            % 可选项、如不需要可删除本行
\phone[mobile]{159~9426~2277}              % 可选项、如不需要可删除本行
%\phone[fixed]{+2~(345)~678~901}               % 可选项、如不需要可删除本行
%\phone[fax]{+3~(456)~789~012}                 % 可选项、如不需要可删除本行
\email{edward.chend@gmail.com}                    % 可选项、如不需要可删除本行
\homepage{http://laodeng.github.com}                  % 可选项、如不需要可删除本行
\extrainfo{男 1990.7}                 % 可选项、如不需要可删除本行
\photo[64pt][0.4pt]{/home/laodeng/Document/TeX/photo_new.jpg}                  % ‘64pt’是图片必须压缩至的高度、‘0.4pt‘是图片边框的宽度 (如不需要可调节至0pt)、’picture‘ 是图片文件的名字;可选项、如不需要可删除本行
%\quote{擅长文献转换为产品。实习时间:2014.5-2014.9}                          % 可选项、如不需要可删除本行

% 显示索引号;仅用于在简历中使用了引言
%\makeatletter
%\renewcommand*{\bibliographyitemlabel}{\@biblabel{\arabic{enumiv}}}
%\makeatother

% 分类索引
%\usepackage{multibib}
%\newcites{book,misc}{{Books},{Others}}
%----------------------------------------------------------------------------------
%            内容
%----------------------------------------------------------------------------------
\begin{document}
\begin{CJK}{UTF8}{gbsn}                       % 详情参阅CJK文件包
\maketitle

\section{教育背景}
\cventry{2009年 -- 2013年}{学士}{中国地质大学(武汉)}{通信工程}{}{}  % 第3到第6编码可留白
\cventry{2013年 -- 2015年}{硕士}{中国地质大学(武汉)}{电子与通信工程}{}{}

%\section{毕业论文}
%\cvitem{题目}{\emph{题目}}
%\cvitem{导师}{导师}
%\cvitem{说明}{\small 论文简介}

\section{工作经验}

\cventry{2013.01 -- 2014.06}{轴承振动检测系统}{}{}{}{
滚动轴承故障的自动检测系统。%
\begin{itemize}%
\item OMRON PLC自动控制程序设计、线路设计、机柜设计、线路连接调试。
\item 机械部分气路设计、急停保护设计。
\item 嵌入式WinCe、Linux程序设计,主要包括串口控制PLC程序,USB控制采样卡。
\item 信号采集及预处理算法设计,主要包括FFT、FIR滤波器。
\item 基于MATLAB GUI分析、SQL存储数据,信号直接通过MATLAB进行算法处理。
\end{itemize}}

\cventry{2014.03 -- 2014.08}{镇海炼化仪表配电箱设计安装}{}{}{}{
镇海炼化部分仪表24V配电箱设计安装。%
\begin{itemize}%
\item PHOENIX 40A 24V电源供电回路设计、主备通路设计。
\item 线路线槽安装,电缆标号设计打印裁剪安装。
\item 空气开关、漏电保护、保险丝回路设计安装。
\end{itemize}}

\cventry{2013.09 -- 2014.04}{SerDes接口实现}{}{}{}{
SerDes接口是一种新的串行高速接口,常用于高端模数转换芯片。速率可达10Gbps。%
\begin{itemize}%
\item JESD204B协议理解及核心内容提炼,全面掌握接口的工作流程,包括硬件接口、成帧方式、编解码器、差错控制等。
\item CML电路设计,通过对Cadence Virtuoso的全定制芯片设计工具的使用,来完成高速接口电路设计。
\item 基于JESD204B协议的8B/10B编解码器设计,参考802.3协议,利用Verilog语言实现。
\end{itemize}}


\cventry{2013.01 -- 2014.01}{餐厨垃圾焚烧设备组装调试维修}{}{}{}{
主要负责船用垃圾焚烧设备控制器的组装、调试工作。%
\begin{itemize}%
\item 贴片封装元器件电路焊接、检查、维修。
\item 上电测试,检查故障,保障安全使用。
\item MSP430程序烧写调试。
\end{itemize}}


\section{工作经验证明}

\quad\quad 兹证明\emph{陈登}(性别:男 身份证:330211199007270052)于2012年8月1日至2014年8月1日在宁波和路测控有限公司工作2年。 
\newline	 
特此证明
\newline
\newline
 \rightline{单位(盖章):宁波和路测控有限公司}
\newline
 \rightline{2014.8.1}

%\section{计算机技能}
%\cvdoubleitem{编程}{C, C++, ASM}{脚本}{Python, Bash, Makefile}
%\cvdoubleitem{硬件}{Verilog, PLC}{仿真}{Matlab, Modelsim}
%\cvdoubleitem{排版}{\LaTeX, Word, Markdown}{系统}{Windows, Ubuntu, CentOS}

%\section{资格证书}
%\cvdoubleitem{CET-4}{510}{CET-6}{480}
%\cvdoubleitem{IELTS}{6.0}{普通话}{二甲}
%\cvdoubleitem{计算机二级}{C程序语言}{计算机三级}{网络技术}

%\section{社会工作}
%\cvitemwithcomment{院学生会办公室主任}{负责学生会年鉴制作,院内学生会议安排。}{2011.05 -- 2012.05}
%\cvitemwithcomment{副班长}{班级评奖、评优,活动组织。}{2010.09 -- 2011.06}

%\section{兴趣爱好}
%\cvitem{骑行}{\small 2012年完成川藏骑行;2013年环青海湖。}
%\cvitem{建站}{\small 掌握建站基本原理,利用Jekyll在GitHub上托管博客。}
%\cvitem{MOOC}{\small 2013年完成5门课程学习。}

%\section{其他 1}
%\cvlistitem{项目 1}
%\cvlistitem{项目 2}
%\cvlistitem{项目 3}
%
%\renewcommand{\listitemsymbol}{-}             % 改变列表符号
%
%\section{其他 2}
%\cvlistdoubleitem{项目 1}{项目 4}
%\cvlistdoubleitem{项目 3}{}

% 来自BibTeX文件但不使用multibib包的出版物
%\renewcommand*{\bibliographyitemlabel}{\@biblabel{\arabic{enumiv}}}% BibTeX的数字标签
%\nocite{*}
%\bibliographystyle{plain}
%\bibliography{publications}                    % 'publications' 是BibTeX文件的文件名

% 来自BibTeX文件并使用multibib包的出版物
%\section{出版物}
%\nocitebook{book1,book2}
%\bibliographystylebook{plain}
%\bibliographybook{publications}               % 'publications' 是BibTeX文件的文件名
%\nocitemisc{misc1,misc2,misc3}
%\bibliographystylemisc{plain}
%\bibliographymisc{publications}               % 'publications' 是BibTeX文件的文件名

\clearpage\end{CJK}
\end{document}


%% 文件结尾 `template-zh.tex'.
