%% start of file `template-zh.tex'.
%% Copyright 2006-2013 Xavier Danaux (xdanaux@gmail.com).
%
% This work may be distributed and/or modified under the
% conditions of the LaTeX Project Public License version 1.3c,
% available at http://www.latex-project.org/lppl/.


\documentclass[12pt,a4paper,sans]{moderncv}   % possible options include font size ('10pt', '11pt' and '12pt'), paper size ('a4paper', 'letterpaper', 'a5paper', 'legalpaper', 'executivepaper' and 'landscape') and font family ('sans' and 'roman')

% moderncv 主题
\moderncvstyle{banking}                        % 选项参数是 ‘casual’, ‘classic’, ‘oldstyle’ 和 ’banking’
\moderncvcolor{green}                          % 选项参数是 ‘blue’ (默认)、‘orange’、‘green’、‘red’、‘purple’ 和 ‘grey’
%\nopagenumbers{}                             % 消除注释以取消自动页码生成功能

% 字符编码
%\usepackage[utf8]{inputenc}                   % 替换你正在使用的编码
%\usepackage{CJKutf8}



% 调整页面出血
\usepackage[scale=0.9]{geometry}
%\setlength{\hintsc对信号进行FIR滤波、FFT变化、Hilbert-Huang变换等olumnwidth}{3cm}           % 如果你希望改变日期栏的宽度

% 自定义命令
\newcommand{\bigquad}{\qquad\qquad\qquad}

% 个人信息
\name{\emph{Chen}}{\emph{Deng}}
%\title{简历题目 (可选项)}                     % 可选项、如不需要可删除本行
%\address{中国地质大学(武汉)}{}            % 可选项、如不需要可删除本行
\phone[mobile]{159~9426~2277}              % 可选项、如不需要可删除本行
%\phone[fixed]{+2~(345)~678~901}               % 可选项、如不需要可删除本行
%\phone[fax]{+3~(456)~789~012}                 % 可选项、如不需要可删除本行
\email{edward.chend@gmail.com}                    % 可选项、如不需要可删除本行
\homepage{laodeng.github.com}                  % 可选项、如不需要可删除本行
\extrainfo{Power by \LaTeX}                 % 可选项、如不需要可删除本行
%\photo[64pt][0.4pt]{/home/laodeng/Document/TeX/photo_new.jpg}                  % ‘64pt’是图片必须压缩至的高度、‘0.4pt‘是图片边框的宽度 (如不需要可调节至0pt)、’picture‘ 是图片文件的名字;可选项、如不需要可删除本行
%\quote{擅长文献转换为产品。实习时间:2014.5-2014.9}                          % 可选项、如不需要可删除本行

% 显示索引号;仅用于在简历中使用了引言
%\makeatletter
%\renewcommand*{\bibliographyitemlabel}{\@biblabel{\arabic{enumiv}}}
%\makeatother

% 分类索引
%\usepackage{multibib}
%\newcites{book,misc}{{Books},{Others}}
%----------------------------------------------------------------------------------
%            内容
%----------------------------------------------------------------------------------
\begin{document}
%\begin{CJK}{UTF8}{gbsn}                       % 详情参阅CJK文件包
\maketitle

\section{Education}
\cventry{2009 -- 2013}{Bachelor}{China University of Geosciences(Wuhan)}{Communication Engineering}{Faculty of Mechanical and Electronic Information}{}  % 第3到第6编码可留白
\cventry{2013 -- 2015}{Master}{China University of Geosciences(Wuhan)}{Electronics and Communications Engineering}{Faculty of Mechanical and Electronic Information}{}

%\section{毕业论文}
%\cvitem{题目}{\emph{题目}}
%\cvitem{导师}{导师}
%\cvitem{说明}{\small 论文简介}

\section{Projects}

\cventry{2013.09 -- 2014.04}{SerDes Receiver}{}{}{}{SerDes interface is a new high-speed serial interfaces, commonly used in analog-digital conversion chip. Rates up to 10Gbps. 
The main design using Verilog language, Modelsim simulation, Design Compiler synthesis. Completed the work of the link layer and transport layer in  receiver and published journal articles.%
\begin{itemize}%
\item Platform:Windows + CentOS + Sublime
\item Tools:Verilog + Modelsim + Design Compiler 
\item Duty:
	\begin{itemize}
	\item Protocol: JESD204B protocol understanding and refining the core master workflow interface.
	\item Data Link Layer: Designed 8B/10B Decoder, descrambler, code group synchronization, frame synchronization, lane synchronization and error control.
	\item Transport Layer: Designed deframer, control interface and error control.
	\end{itemize}
\end{itemize}}
\cventry{2010.07 -- 2014.08}{Bearing vibration detection system}{}{}{}{
Automatic rolling bearing fault detection system. Expected to change the existing small bearing human auditory detection technology as the basis to judge the status quo.%
\begin{itemize}%
\item Hardware:Raspberry Pi + Omron PLC + MSP430F149 + FT232H
\item Software:Eclipse + Ubuntu + Raspbian + MySQL + CCS
\item Tools:C++/C + Python + JavaFX + Matlab
\item Duty:
	\begin{itemize}
	\item Signal: Based on MSP430F149 and FT232H constitute sample card. It can flexibly change the length of the sampling frequency and data via configuration interface.
	\item Preprocess: The sampled signal level conversion, FIR filters, FFT, Hilbert-Huang transform and denoising work.
	\item Data: Matlab GUI design data capture interface, automatically or manually collect data , labeled it and stored the data in database.
	\item Algorithms: preprocessing the signal, based on large amounts of data, the use of learning algorithm to classify the decision.
	\item Implementation: Embedded Linux programming, including serial PLC program control, USB sample card program, the control flow of the program, UI program.
	\item Control: PLC to control the mechanical parts, take PLC status by C Command mode.
	\end{itemize}
\end{itemize}}

\section{Work}

\cventry{2010.07 -- 2014.08}{Ningbo HOLO Instruments}{}{}{}{
Ningbo HOLO Instruments is one of the earliest in the condition monitoring system development and application of professional firms.%
\begin{itemize}%
\item Maintenance of existing projects, Sinopec Zhenhai Refining \& Chemical Plant dynamic monitoring platform fault monitoring and maintenance.
\item Development of new projects, bearing vibration measurement system design and development, USB sampling board design and development.
\item Product manufacturing, food waste incineration controller shipboard production.
\end{itemize}}

\section{Social}

\cvitemwithcomment{Student Union Office}{Responsible for student yearbook production.}{2011.05 -- 2012.05}
\cvitemwithcomment{Vice squad}{Class awards, activities of the organization.}{2010.09 -- 2011.06}

\section{Computer Skills}

\subsection{Programming}
\cvdoubleitem{C/C++}{\hfill 4 year\qquad Skilled\bigquad}{Verilog}{\hfill 2 year\qquad Skilled\bigquad}
\cvdoubleitem{Python}{\hfill 2 year\qquad General\bigquad}{Bash}{\hfill 1 year\qquad General\bigquad}
\subsection{Tools}
\cvdoubleitem{Matlab}{\hfill 4 year\qquad Skilled\bigquad}{Modelsim}{\hfill 2 year\qquad Skilled\bigquad}
\cvdoubleitem{Git}{\hfill 2 year\qquad General\bigquad}{Design Compiler}{\hfill 1 year\qquad General\bigquad}
\cvdoubleitem{Word}{\hfill 5 year\qquad Skilled\bigquad}{\LaTeX}{\hfill 2 year\qquad Skilled\bigquad}
\cvdoubleitem{Vim}{\hfill 2 year\qquad General\bigquad}{PLC}{\hfill 1 year\qquad Skilled\bigquad}
\subsection{OS}
\cvdoubleitem{Windows}{\hfill 8 year\qquad Skilled\bigquad}{Ubuntu}{\hfill 2 year\qquad Skilled\bigquad}
\cvdoubleitem{Raspbian}{\hfill 1 year\qquad Skilled\bigquad}{CentOS}{\hfill 1 year\qquad General\bigquad}
\subsection{Datanase}
\cvdoubleitem{MySQL}{\hfill 1 year\qquad General\bigquad}{SQL Server}{\hfill 1 year\qquad Skilled\bigquad}

\section{Certificates}

\subsection{Language}
\cvdoubleitem{CET-4}{510}{CET-6}{485}
\cvdoubleitem{IELTS}{6.0}{}{}
\subsection{Computer}
\cvdoubleitem{NCER}{C programming language}{NCER}{Network Technology}

\section{Interests}

\cvitem{Riding}{\small 2012 completion of the Sichuan-Tibet riding; 2013 Qinghai Lake.}
\cvitem{Website}{\small Master station basic principle, the use Jekyll hosted blog on GitHub; build GitLab server within the network.}
\cvitem{MOOC}{\small 2013 to complete five courses; 2014 currently completed a course of study.}
\cvitem{Rock}{\small There used to go LiveHouse listen to live music; went to Beijing strawberries, Midi Music Festival in 2011 during May.}
\cvitem{Travel}{\small 2013 Thailand; 2013 Malaysia. .}

%\section{其他 1}
%\cvlistitem{项目 1}
%\cvlistitem{项目 2}
%\cvlistitem{项目 3}
%
%\renewcommand{\listitemsymbol}{-}             % 改变列表符号
%
%\section{其他 2}
%\cvlistdoubleitem{项目 1}{项目 4}
%\cvlistdoubleitem{项目 3}{}

% 来自BibTeX文件但不使用multibib包的出版物
%\renewcommand*{\bibliographyitemlabel}{\@biblabel{\arabic{enumiv}}}% BibTeX的数字标签
%\nocite{*}
%\bibliographystyle{plain}
%\bibliography{publications}                    % 'publications' 是BibTeX文件的文件名

% 来自BibTeX文件并使用multibib包的出版物
%\section{出版物}
%\nocitebook{book1,book2}
%\bibliographystylebook{plain}
%\bibliographybook{publications}               % 'publications' 是BibTeX文件的文件名
%\nocitemisc{misc1,misc2,misc3}
%\bibliographystylemisc{plain}
%\bibliographymisc{publications}               % 'publications' 是BibTeX文件的文件名

\clearpage
%\end{CJK}
\end{document}


%% 文件结尾 `template-zh.tex'.
